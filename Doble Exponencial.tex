%%
% Modificación de una plantilla de Latex para adaptarla al castellano.
%%

%%%%%%%%%%%%%%%%%%%%%
% Thin Sectioned Essay
% LaTeX Template
% Version 1.0 (3/8/13)
%
% This template has been downloaded from:
% http://www.LaTeXTemplates.com
%
% Original Author:
% Nicolas Diaz (nsdiaz@uc.cl) with extensive modifications by:
% Vel (vel@latextemplates.com)
% 
% Copied from https://github.com/Lothar94/Poisson
%
% License:
% CC BY-NC-SA 3.0 (http://creativecommons.org/licenses/by-nc-sa/3.0/)
%
%%%%%%%%%%%%%%%%%%%%%

%----------------------------------------------------------------------------------------
%	PACKAGES AND OTHER DOCUMENT CONFIGURATIONS
%----------------------------------------------------------------------------------------

\documentclass[a4paper, 10pt]{article} % Font size (can be 10pt, 11pt or 12pt) and paper size (remove a4paper for US letter paper)
\usepackage{helvet}
\renewcommand{\familydefault}{\sfdefault}
\usepackage[protrusion=true,expansion=true]{microtype} % Better typography
\usepackage[usenames,dvipsnames]{color} % Coloring code
\usepackage[utf8]{inputenc}
\usepackage{enumerate}
\usepackage{enumitem}

% Márgenes
\usepackage{anysize}
\marginsize{3cm}{3cm}{2.5cm}{2.5cm}

% Imágenes
\usepackage{graphicx} 
\usepackage{wrapfig} % Allows in-line images

% sudo apt-get install texlive-lang-spanish
\usepackage[spanish]{babel} % English language/hyphenation
\selectlanguage{spanish}
% Hay que pelearse con babel-spanish para el alineamiento del punto decimal
\decimalpoint
\usepackage{dcolumn}
\newcolumntype{d}[1]{D{.}{\esperiod}{#1}}
\makeatletter
\addto\shorthandsspanish{\let\esperiod\es@period@code}
\makeatother

% Símbolos matemáticos
\usepackage{amsmath}
\usepackage{amssymb}
\usepackage{accents}
\let\oldemptyset\emptyset
\let\emptyset\varnothing

% Hipervínculos
\usepackage[hidelinks]{hyperref}

\usepackage[section]{placeins} % Para gráficas en su sección.
\usepackage[T1]{fontenc} % Required for accented characters
\newenvironment{allintypewriter}{\ttfamily}{\par}
\setlength{\parindent}{0pt}
\parskip=8pt
\linespread{1.05} % Change line spacing here, Palatino benefits from a slight increase by default

% Contadores
\newcounter{def}
\newcounter{teo}

\makeatletter

\renewcommand{\maketitle}{ % Customize the title - do not edit title and author name here, see the TITLE block below
\begin{center} % Center align
{\Huge\@title} % Increase the font size of the title
\end{center}

\vspace{20pt} % Some vertical space between the title and author name

\begin{flushright} % Right align
{\large\@author} % Author name
\\\@date % Date

\vspace{40pt} % Some vertical space between the author block and abstract
\end{flushright}
\renewcommand{\baselinestretch}{0.5}

}
%----------------------------------------------------------------------------------------
%	TITLE
%----------------------------------------------------------------------------------------

\title{\textbf{Distribución Doble Exponencial}\\ % Title
\vspace{20 pt}
} % Subtitle

\author{\textsc{Óscar Bermúdez Garrido\\José Carlos Entrena Jiménez} % Author
\\{\textit{Universidad de Granada}}} % Institution

\date{\today} % Date

%----------------------------------------------------------------------------------------

\begin{document}
\maketitle
\tableofcontents
\pagebreak

\section{Introducción}

La distribución doble exponencial, también conocida como distribución de Laplace en honor al matemático Pierre-Simon Laplace, es una densidad de probabilidad continua con dominio en la recta real. Se obtiene reflejando la distribución exponencial alrededor de su media. Su función de densidad se asemeja la de la distribución normal, pero como considera el valor absoluto y no el cuadrado, las colas de la distribución irán a $0$ más lentamente. 

\section{Función de densidad}

La función de densidad de la distribución doble exponencial es la siguiente: 

$$L(x|\mu, b)=\frac{1}{2b}e^\frac{-|x-\mu|}{b}$$

En esta distribución, $\mu$ es un parámetro de localización, mientras que $b$ es un parámetro de escala. 

Pese a sus similaridades con la distribución normal, debido al uso del valor absoluto, esta función de distribución no tiene forma acampanada, sino que presenta un pico, es decir, un punto en el que no se puede diferenciar, $x = \mu$, factor importante a tener en cuenta a la hora de tratar con esta función analíticamente. De igual forma, a la hora de integrar, dividiremos $\mathbb{R}$ en las dos semirrectas $\left(-\infty, \mu\right], \left[\mu, \infty\right)$.

Con esta idea, veamos que la integral en la recta real de la función de densidad es $1$: 

$$\int_{-\infty}^{\infty} L(x|\mu, b)dx = \int_{-\infty}^{\infty} \frac{1}{2b}e^\frac{-|x-\mu|}{b}dx =
\frac{1}{2b}\int_{-\infty}^{\infty}e^\frac{-|x-\mu|}{b}dx = 
\frac{1}{2b}\left( \int_{-\infty}^{\mu}e^\frac{-\mu+x}{b}dx + \int_{\mu}^{\infty}e^\frac{-x+\mu}{b}dx \right) =
\frac{1}{2b}2b = 1$$
 

\section{Función generatriz de momentos}


La función generatriz de momentos se define como sigue:
$$\phi(t)=E[e^{tX}], \quad t \in \left(\frac{-1}{b}, \frac{1}{b} \right)$$

Que en nuestro caso da lugar a:\\
% Justificación encontrada en los puntos 7 y 24 de http://www.math.uah.edu/stat/special/Laplace.html
$\displaystyle \phi(t) = \int_{-\infty}^{\infty} e^{tx}\frac{1}{2b}e^\frac{-|x-\mu|}{b}dx = 
\frac{1}{2b}\left( \int_{-\infty}^{\mu} e^{tx} e^\frac{-\mu+x}{b}dx + \int_{\mu}^{\infty} e^{tx} e^\frac{-x+\mu}{b}dx \right) =
\frac{1}{2b} \bigg( \int_{-\infty}^{\mu} e^\frac{-\mu+x(bt+1)}{b}dx + \int_{\mu}^{\infty} e^\frac{\mu+x(bt-1)}{b}dx \bigg) =
\frac{1}{2b} e^{\mu t} \left(\frac{b}{bt+1} - \frac{b}{bt-1} \right) = 
\frac{1}{2b} e^{\mu t} \left(\frac{-2 b}{b^2t^2-1} \right) = \frac {e^{t\mu}} {1-b^2t^2}$

\subsection{Esperanza}

Para calcular la esperanza de la función de distribución, tomamos el momento de orden 1:

$$\frac{\partial\phi}{\partial t} = \frac{e^{t\mu} (-t^2 b^2\mu + \mu + 2b^2t)}{(1-b^2 t^2)^2}$$

Con $t = 0$, se tiene: 

$$E[X]=\mu$$

\subsection{Varianza}

La varianza se puede expresar como $\sigma^2 = E[X^2] - E[X]^2$. Conocido el momento de orden 1, calculamos el de orden 2 usando
la función generatriz de momentos:

$$\frac{\partial^2\phi}{\partial t^2} = -\frac{e^{t\mu}(\mu^2(b^2 t^2 -1)^2 -4\mu b^2 t(b^2 t^2 -1)+2b^2(3b^2 t^2+1))}{(b^2 t^2 -1)^3}$$

Evaluando en $t = 0$ tenemos

$$E[X^2]=2b^2 + \mu^2$$

Y la expresión de la varianza queda de la siguiente forma:

$$\sigma^2=2b^2 + \mu^2 - \mu^2=2b^2$$

\section{Estimador máximo verosimil}

\subsection{Insesgadez}
\addtocounter{def}{1}
\emph{Definición \arabic{def}}: Se denomina sesgo de un estimador a la diferencia entre la esperanza del estimador
y el verdadero valor del parámetro a estimar. Diremos que un estimador es insesgado si su sesgo es nulo, por ser
su esperanza igual al parámetro que se desea estimar.

\subsection{Eficiencia}
\addtocounter{def}{1}
\emph{Definición \arabic{def}}: Un estimador $\hat{\theta}_1$ se dice que es más eficiente que otro estimador
$\hat{\theta}_2$, si la varianza del primero es menor que la del segundo, esto es,  $Var(\hat{\theta}_1)<Var(\hat{\theta}_2)$.

\subsection{Consistencia}
\addtocounter{teo}{1}
\emph{Teorema \arabic{teo}}: Una condición suficiente para que $\hat{\theta}$ sea un estimador consistente es que
dicho estimador tiene que verificar las dos condiciones que siguen:

\begin{description}
	\item $E[\hat{\theta}] \rightarrow \theta$
	\item $Var(\hat{\theta}) \rightarrow 0$
\end{description}

Cuando $n \rightarrow \infty$.

\subsection{Suficiencia}
\emph{Teorema \arabic{teo}}: Sean $X_1, X_2, ..., X_n$ variables aleatorias independientes con una función
distribución de distribución conjunta $f(x_1, x_2, ..., x_n | \theta)$ que depende del parámetro $\theta$.

Entonces se dice que el estadístico $u(x_1, ..., x_n)$ es suficiente para $\theta$ si y solamente si
$f(x_1, x_2,\dots, x_n | \theta)$ se puede factorizar de la siguiente forma:

$$ f(x_1,...,x_n| \theta) = \Phi(u(x_1, ..., x_n) | \theta)\cdot h(x_1, ..., x_n) $$

\end{document}
