%%
% Modificación de una plantilla de Latex para adaptarla al castellano.
%%

%%%%%%%%%%%%%%%%%%%%%
% Thin Sectioned Essay
% LaTeX Template
% Version 1.0 (3/8/13)
%
% This template has been downloaded from:
% http://www.LaTeXTemplates.com
%
% Original Author:
% Nicolas Diaz (nsdiaz@uc.cl) with extensive modifications by:
% Vel (vel@latextemplates.com)
% 
% Copied from https://github.com/Lothar94/Poisson
%
% License:
% CC BY-NC-SA 3.0 (http://creativecommons.org/licenses/by-nc-sa/3.0/)
%
%%%%%%%%%%%%%%%%%%%%%

%----------------------------------------------------------------------------------------
%	PACKAGES AND OTHER DOCUMENT CONFIGURATIONS
%----------------------------------------------------------------------------------------

\documentclass[a4paper, 10pt]{article} % Font size (can be 10pt, 11pt or 12pt) and paper size (remove a4paper for US letter paper)
\usepackage{helvet}
\renewcommand{\familydefault}{\sfdefault}
\usepackage[protrusion=true,expansion=true]{microtype} % Better typography
\usepackage{graphicx} % Required for including pictures
\usepackage[usenames,dvipsnames]{color} % Coloring code
\usepackage{wrapfig} % Allows in-line images
\usepackage[utf8]{inputenc}
\usepackage{enumerate}
\usepackage{enumitem}

% Imágenes
\usepackage{graphicx} 

\usepackage{amsmath}
% para importar svg
%\usepackage[generate=all]{svgfig}

% sudo apt-get install texlive-lang-spanish
\usepackage[spanish]{babel} % English language/hyphenation
\selectlanguage{spanish}
% Hay que pelearse con babel-spanish para el alineamiento del punto decimal
\decimalpoint
\usepackage{dcolumn}
\newcolumntype{d}[1]{D{.}{\esperiod}{#1}}
\makeatletter
\addto\shorthandsspanish{\let\esperiod\es@period@code}
\makeatother

\usepackage{longtable}
\usepackage{tabu}
\usepackage{supertabular}

\usepackage{multicol}
\newsavebox\ltmcbox

% Install texlive-science
% Para algoritmos
\usepackage{algorithm}
\usepackage{algorithmic}
\usepackage{amsthm}

% Para matrices
\usepackage{amsmath}

% Símbolos matemáticos
\usepackage{amssymb}
\usepackage{accents}
\let\oldemptyset\emptyset
\let\emptyset\varnothing

\usepackage[hidelinks]{hyperref}

\usepackage[section]{placeins} % Para gráficas en su sección.
\usepackage[T1]{fontenc} % Required for accented characters
\usepackage{tikz}
\newenvironment{allintypewriter}{\ttfamily}{\par}
\setlength{\parindent}{0pt}
\parskip=8pt
\linespread{1.05} % Change line spacing here, Palatino benefits from a slight increase by default

\makeatletter
\renewcommand\@biblabel[1]{\textbf{#1.}} % Change the square brackets for each bibliography item from '[1]' to '1.'
\renewcommand{\@listI}{\itemsep=0pt} % Reduce the space between items in the itemize and enumerate environments and the bibliography
\newcommand{\imagen}[2]{\begin{center} \includegraphics[width=90mm]{#1} \\#2 \end{center}}
\newcommand{\RFC}[1]{\href{https://www.ietf.org/rfc/rfc#1.txt}{RFC-#1}}

\renewcommand{\maketitle}{ % Customize the title - do not edit title and author name here, see the TITLE block below
\begin{center} % Center align
{\Huge\@title} % Increase the font size of the title
\end{center}

\vspace{20pt} % Some vertical space between the title and author name

\begin{flushright} % Right align
{\large\@author} % Author name
\\\@date % Date

\vspace{40pt} % Some vertical space between the author block and abstract
\end{flushright}
\renewcommand{\baselinestretch}{0.5}

}
%----------------------------------------------------------------------------------------
%	TITLE
%----------------------------------------------------------------------------------------

\title{\textbf{Distribución Doble Exponencial}\\ % Title
\vspace{20 pt}
} % Subtitle

\author{\textsc{Óscar Bermúdez Garrido\\José Carlos Entrena Jiménez} % Author
\\{\textit{Universidad de Granada}}} % Institution

\date{\today} % Date

%----------------------------------------------------------------------------------------
%\setcounter{secnumdepth}{3}
\usepackage{anysize}
\marginsize{3cm}{3cm}{2.5cm}{2.5cm}

\newcounter{def}
\newcounter{teo}


\begin{document}
\maketitle
\tableofcontents
\setcounter{page}{1}
\pagebreak

\section{Introducción}

La distribución doble exponencial, también conocida como distribución de Laplace en honor al matemático Pierre-Simon Laplace, es una densidad de probabilidad continua con dominio en la recta real. 

\section{Función de densidad}

$$L(x|\mu, b)=\frac{1}{2b}e^\frac{-|x-\mu|}{b}$$

En esta distribución, $\mu$ es un parámetro de localización, mientras que $b$ es un parámetro de escala. 

Podemos ver que la integral en la recta real de la función de densidad es 1: 

$$\int_{-\infty}^{\infty} L(x|\mu, b)dx = \int_{-\infty}^{\infty} \frac{1}{2b}e^\frac{-|x-\mu|}{b}dx =
\frac{1}{2b}\int_{-\infty}^{\infty}e^\frac{-|x-\mu|}{b}dx = 
\frac{1}{2b}\left( \int_{-\infty}^{\mu}e^\frac{-\mu+x}{b}dx + \int_{\mu}^{\infty}e^\frac{-x+\mu}{b}dx \right) =
\frac{1}{2b}2b = 1$$

\section{Función generatriz de momentos}


La función generatriz de momentos se define como sigue:
$$\phi(t)=E[e^{tX}]$$

Que en nuestro caso da lugar a:\\
$\displaystyle \phi(t) = E[e^{tx}] = \int_{-\infty}^{\infty} e^{tx}L(x|\mu, b)dx =
%\int_{-\infty}^{\infty} e^{tx}\frac{1}{2b}e^\frac{-|x-\mu|}{b}dx = 
%\frac{1}{2b}\int_{-\infty}^{\infty} e^{tx}e^\frac{-|x-\mu|}{b}dx = \\
%\frac{1}{2b}\left( \int_{-\infty}^{\mu} e^{tx} e^\frac{-\mu+x}{b}dx + \int_{\mu}^{\infty} e^{tx} e^\frac{-x+\mu}{b}dx \right) =
%\frac{1}{2b}2b =
% Hay una justificación en los puntos 8 y 24 de http://www.math.uah.edu/stat/special/Laplace.html
[\cdots] = \frac {e^{t\mu}} {1-b^2t^2}$

\subsection{Esperanza}

Para calcular la esperanza de la función de distribución, tomamos el momento de orden 1 y evaluamos en $t = 0$. 

$$\frac{\partial\phi}{\partial t} = \frac{e^{t\mu} (-t^2 b^2\mu + \mu + 2b^2t)}{1-b^2 t^2}$$
Con $t = 0$, se tiene: 
$$E[X]=\mu$$

\subsection{Varianza}

La varianza se puede expresar como $\sigma^2 = E[X^2] - E[X]^2$ por lo que es tan solo calcular el momento de
orden 2 usando la función generatriz de momentos, evaluamos la expresión en t = 0 y tenemos que:

\section{Estimador máximo verosimil}

\subsection{Insesgadez}
\addtocounter{def}{1}
\emph{Definición \arabic{def}}: Se denomina sesgo de un estimador a la diferencia entre la esperanza del estimador
y el verdadero valor del parámetro a estimar. Diremos que un estimador es insesgado si su sesgo es nulo, por ser
su esperanza igual al parámetro que se desea estimar.

\subsection{Eficiencia}
\addtocounter{def}{1}
\emph{Definición \arabic{def}}: Un estimador $\hat{\theta}_1$ se dice que es más eficiente que otro estimador
$\hat{\theta}_2$, si la varianza del primero es menor que la del segundo, esto es,  $Var(\hat{\theta}_1)<Var(\hat{\theta}_2)$.

\subsection{Consistencia}
\addtocounter{teo}{1}
\emph{Teorema \arabic{teo}}: Una condición suficiente para que $\hat{\theta}$ sea un estimador consistente es que
dicho estimador tiene que verificar las dos condiciones que siguen:

\begin{description}
\item $E[\hat{\theta}] \rightarrow \theta$
\item $Var(\hat{\theta}) \rightarrow 0$
\end{description}
Cuando $n \rightarrow \infty$.

\subsection{Suficiencia}

\end{document}
